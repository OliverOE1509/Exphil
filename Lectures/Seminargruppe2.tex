\documentclass[11pt, a4paper]{article}

% ===== ESSENTIAL PACKAGES FOR PHILOSOPHY =====
\usepackage[english]{babel} % For hyphenation and linguistic rules
\usepackage{geometry} % For comfortable margins for reading
\geometry{a4paper, margin=1.2in} % Slightly larger margins for notes
\usepackage{setspace} % For line spacing
\onehalfspacing % 1.5 line spacing for readability

\usepackage[style=apa, backend=biber]{biblatex} % For APA-style citations (great for philosophy)
\addbibresource{references.bib} % Your BibTeX file

% ===== PACKAGES FOR LOGIC & ARGUMENT STRUCTURE =====
\usepackage{logicproof} % For formal logic proofs (optional)
\usepackage{enumitem} % For customizing lists

% ===== CUSTOM ENVIRONMENTS FOR PHILOSOPHICAL ARGUMENTS =====
\newenvironment{argument}{\begin{quote}\itshape}{\end{quote}} % For quoting arguments
\newenvironment{objection}{\begin{quote}\itshape\textbf{Objection: }}{\end{quote}} % For objections
\newenvironment{reply}{\begin{quote}\itshape\textbf{Reply: }}{\end{quote}} % For replies

% ===== DOCUMENT BEGIN =====
\title{Seminar gruppe 2 (117)}
\author{Oliver Ekeberg}
\date{}

\begin{document}
\maketitle

\tableofcontents

% No abstract is typical for shorter philosophy essays

\section{Introduction}
\label{sec:introduction}


Tobakksreklamer bør forbys fordi de er medvirkende til at unge begynner å røyke. Og selvom tobakksreklamer ikke hadde
hatt en slik effekt, burde de fortsatt forbys fordi de gjør at folk som
allerede røyker, får feil inntrykk av at røyking er akseptabelt.






\section{Critical Discussion}
\label{sec:critical}

This is where you present your original contribution—your critique, analysis, or positive view.

\begin{objection}
    However, one might object that the first premise is false. What if genetic engineering allows some humans to become immortal? This would undermine the entire argument.
\end{objection}

\begin{reply}
    In reply to this objection, we can modify the first premise to be \textit{All biologically natural men are mortal}. This accommodates the objector's point without surrendering the argument's validity.
\end{reply}

You can also use footnotes for tangential but interesting points.\footnote{This is a great place to mention an interesting aside, a further reference, or a complicating point that would break the flow of the main argument.}

\section{Conclusion}
\label{sec:conclusion}

Briefly summarize the path your paper has taken and restate your conclusion. Avoid introducing new arguments or evidence here.

% ===== BIBLIOGRAPHY =====
% Print the bibliography
\printbibliography

\end{document}