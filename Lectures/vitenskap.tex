\documentclass[11pt]{article}
\usepackage[a4paper,margin=1.8cm]{geometry}
\usepackage{amsmath, amssymb, amsthm}  % Essential math packages
\usepackage{graphicx}                   % For figures
\usepackage{hyperref}                   % Clickable links
\usepackage{parskip} % cleaner spacing
\usepackage{enumitem}

\newtheorem{definition}{Definisjon}

\title{Forelesning 3, kap 3, Hvordan skille  vitenskap fra pseudovitenskap}
\author{Your Name}
\date{\today}

\begin{document}
\maketitle


\tableofcontents

Overview

\begin{itemize}
    \item Hvordan se forskjell på pseudo og vanlig vitenskap
    \item Hva er vitenskap, og hva gjør en aktivitet vitenskapelig?
    \item Hvorfor er vitenskapelige resultater spesielt pålitelige?
\end{itemize}


For å ta en beslutning, for eksempel å omstille til grønn energi,  så trengs det  kunnskap og eksperter som kan forklare  hva som er best.

Men hvem er ekspertene? Er en astrologi ekspert en som kan mye om stjernetegn? Hvordan vet vi at astrologi er en vitenskap?
Boken sier at en grunn for å si "nei" er at astrologi ikke er en vitenskap. Det er derfor  det ikke undervises i astrologi på uio, og derfor finansdepartementet ikke konsulterer.




Pseudovitenskap er utsatt for å konstrueres av politiske, personlige meninger.


For eksempel:
\begin{itemize}
    \item \textbf{Medisin:} Det finnes medisin som en vitenskap og som en mening: Hva skal et kosthold bestå av? Hva er sunnere av fisk og kjøtt?
    \item Da Galileo kom med sin teori om at jorden ikke var i sentrum av universet, ble han dømt til livsvarig sensur og ble akademisk sensurert.
    \item Heksekraft
\end{itemize}



Hvordan bestemme om en vitenskapelig metode er vitenskapelig?

\textbf{Lakatos}

Lakatos metode for å komme frem til om noe er vitenskapelig er å begynne med å sette opp klare eksempler på det som er vitenskapelig (som Einstein og Newton), og sette det opp mot klare eksempler på pseudovitenskap (astrologi og Heksekraft).

Eksempler som skiller god vitenskapelig praksis mot pseudovitenskapelig praksis:
\begin{enumerate}
    \item Det å bygge en teori med eksperiment og obervasjon er ikke nok grunnlag. Da Anne Pedersdotter ble brent i 1590 i Bergen, var det klare observasjoner fra mange vitner. De var overbevist om at de hadde klare argument. Men det ble bre trukket absurde konklusjoner fra absurde eksperiment basert på absurde bakgrunnsteorier
    \item Om vi krever at vitenskap skal kunne bevises med  100 $\%$ sikkerhet, har vi utelukket  alle  empiriske vitenskaper. Ingen påstander i fysikk, kjemi eller biologi kan bevises på samme måte som i matematikk. Er alltid rom for tvil og videre arbeid i empiriske vitenskap
    \item Karl Popper sa: God vitenskapelig praksis er å vise hvordan man kan falsifisere teorien sin. Falsifisere betyr å vise hvordan den er feil. Ifølge lakatos ville en forsker som jobbet med newtons eller Einsteins toerier, være i like lite stand til å spesifisere et falsifiserende eksperiment som en som arbeidet innen det Lakatos mener er pseudovitenskap
    \item Lakatos støtter heller ikke Kuhn. Kuhn var om paradigmeskifte. Han sammenligner overgangen fra vitensakap til en annen med det å skifte religion eller et irrasjonelt trossystem. Slike skifter er ikke basert på evidens eller overbevisende argument. Lakatos mener overgangen fra Newtons til Einsteins system var rasjonell og basert på solid evidens.
\end{enumerate}


\vspace{1em}
Lakatos mener videre vi bør fokusere på forskningsprogrammer. Det vil si at vitenskapelig prosjekt er en levende aktivitet som forandrer seg over tid.

En viktig del av svaret for lakatos  er at et godt forskningsprogram forutsier uventede pg oppsiktsvekkende fakta eller hendelser.

For eksempel: Halley brukte Newtons teorier for å forutsi mange tiår i forkant hvor det Halleys komet ville vise seg. For Lakatos er Halleys komet en ideell illustrasjon av hvordan gode forskningsprogram funker.

\vspace{1em}
Vi har på den andre siden degenererte forskningsprogram. For eksempel: Astrologi. Det sier at alle som er født en tid på året, vil leve samme skjebne, når og hvordan de vil dø. Hvis en tsunami innslår en by, hvor vi antar alle mennesker har forskjellig fødselsdag, da har alle ledet samme skjebne, og Astrologi er ikke et bra eksempel på et bra forskningsprogram



\vspace{1em}
En annen forklaring er at falsifikasjonsteori som kommer fra Karl Popper, sier at en teori skal avvises om den kan bevises som feil av eksperimentell fakta, og en bedre teori er henvist. Lakatos hevdet at en teori skal vurderes som en del av et større forskningsprogram.



\subsection{Thomas Kuhn, 1922-1996}
Hadde stor tro på paradigmer. Det vil si et distinkt sett som inneholder teorier, tenkemønstre og andre karakteristikker forenlige med en sivilisiasjon. Når det oppstår anomalier, altså hull i paradigmet, så vil et paradigmeskiftet komme til. Noe som kan kalles for en vitenskapelig revolusjon


\subsection{Karl Popper, 1902-1994}
Mente at vitenskapsteorien vår kan aldri fullt ut bevises som 100$\%$ riktig fra empiriske observasjoner, og at man derfor praktiserer bedre vitenskap ved å gå ut ifra falsifikasjon. \textbf{Falsifikasjon} utgjør et demarkasjonsprinsipp som man kan bruke til å skille empirisk vitenskap fra matematikk


\textbf{eksempel:}
\vspace{1em}
En gruppe med 10 svaner $->$ svaner er hvite

Siden det eksisterer svarte svaner (i Australia) så vil observasjonen av en svart svane føle til \textbf{demarkasjon} av teorien. 

\vspace{1em}
Følgelig kan en ros for falsifikasjonsprinsipp være at det fører til en iterativ prosess, hvor vi starter med et utsagn, og justerer teorien vår, til den passer med realiteten.  

\subsection{Lakatos, 1922-1974}

En ungarsk matematiker og filosof som overtalte kjæresten sin til å ta livet sitt. 

\vspace{1em}
Et forskningsprogram er basert på en hard kjerne av teoretiske forutsetninger, som man ikke kan forlate uten å forlate selve programmet i seg selv. Det er dette som er likheten mellom Kuhn og Lakatos. 


\end{document}
