
\documentclass[11pt, a4paper]{article}

% ===== ESSENTIAL PACKAGES =====
\usepackage{geometry}
\geometry{a4paper, margin=1.2in}
\usepackage{setspace}
\onehalfspacing

\usepackage{catchfile}

% ===== WORD COUNT COMMAND =====
\newcommand{\wordcount}{%
    \ifnum\pdf@shellescape=1%
        \immediate\write18{texcount -1 -sum Oblig1.tex > wordcount.txt}%
        \CatchFileDef{\words}{wordcount.txt}{}%
        \words%
    \else%
        [Enable -shell-escape]%
    \fi%
}

% ===== CUSTOM ENVIRONMENTS FOR PHILOSOPHICAL ARGUMENTS =====
\newenvironment{argument}{\begin{quote}\itshape\textbf{Argument: }}{\end{quote}}
\newenvironment{objection}{\begin{quote}\itshape\textbf{Innvending: }}{\end{quote}}
\newenvironment{reply}{\begin{quote}\itshape\textbf{Reply: }}{\end{quote}}

% ===== DOCUMENT BEGIN =====
\title{Kap. 9 - Hva er sinnets plass i naturen?}
\author{Oliver Ekeberg}
\date{\today}



\begin{document}
\maketitle

\tableofcontents


\section{Er jeg en materiell ting?}

I dette kapittelet skal vi undersøke hvordan vi passer inn i verdensbildet naturvitenskapen gir oss. Og særlig: \textbf{\textit{Kropp sinn problemet}}


"Er jeg nok en gjenstand som blyanten og laptopen foran meg, eller er jeg en sjel eller en immateriell gjenstand?"\\

Skal her se på utdrag fra en debatt mellom to 1700-talls filosofer; Descartes og Elisabeth av Bohmen


\begin{itemize}
    \item \textbf{\textit{Descartes: Fremstiller et viktig argument for "Jeg er ikke nok en materiell gjenstand". En av de største forkjempere for dualisme, den oppfatning at sinner er grunnleggende forskjellig fra alle andre materielle gjenstander}}
    \item \textbf{\textit{Bohmen: Fremstiller et viktig motargument for dualisme. Argumenterer for at sinnet i bunn og grunn er en materiell ting. Dette synet kalles materialisme}}
\end{itemize}


\section{Avgjøres saken av hjernevitenskapen?}


Det finnes en vitenskap som studerer sinnet som nok en materiell ting. Vi ser på hjerneceller og hvordan det har en påvirkning på vår oppførsel og oppfatning

\subsection{Matematikken og det mekaniske universet}

\textit{Descartes hadde en fundamental uenighet med Aristoteles, som trodde på at kunnskap begynner fra sansene og sjelen, fordi vi kan gjenkjenne likhet. Descares trodde på matematisk rasjonale og var ikke enig i det teologiske verdenssynet; at alt har en hensikt} \\

Videre hadde både Elisabeth Bohmen og Descartes et mekanistisk syn på den fysiske verden. Det vil si at hele universet er en stor kompleks maskin med kausale effekter som er til en viss grad predikerbar. Descartes utvidet dette til sinnet, og sier at menneskekroppen og hjernen er komplekse systemer. Det er dette Descartes og Elisabeth er uenig i Aristoteles sitt teologiske verdenssyn: \textbf{\textit{At alle ting har en hensikt}}. Fordi mekanistisk syn betyr at ikke alt har en hensikt, men det følger fysiske og naturlover. Altså at verden består av materie i bevegelse, styrt av matematiske lover. Det er ingen indre mål som trengs for å forklare hvorfor regnet har ødelagt avlingen

\textit{Olivers syn på mekanistisk syn: Verden er ikke så enkel slik at den kan brytes ned og generaliseres til likninger. Det er en stor grad av tilfeldighet og sannsynlighet som spiller en rolle for hvordan verden går. Det finnes fenomener vi ikke kan vite om, og derfor ikke kalkulere. Slike hendelser kan vi ikke predikere med sikkerhet.}\\

En annen sammenlikning det er viktig å ta mellom Descartes og Aristoteles, er at

\begin{itemize}
    \item Aristoteles trodde at sjelen og kroppen var inseparabel
    \item Descartes argumenterte for en skarp separasjon mellom ikke-fyfiske sjelen og den fysiske kroppen. (Cogito ergo sum "I think, therefore I am")
\end{itemize}





\textbf{\textit{Phineas Gage}} var en amerikansk jernbanearbider, han bygde damplokomotivspor, hvor han i 1848 fikk en jernstang gjennom hodet sitt. Til tross for at han overlevde og ble frisk, forsvant deler av hjernen hans. Han fikk en annen personlighet, fra å være stort likt av kollegaer, til å bli frekk. Phineas ble brukt i mange eksperimenter på 1900-tallet, til det som i dag kalles nevrovitenskap.\\

Eksempelet om Phineas forklarer hvorfor noen tror at hjernen og sinnet er nært tilknyttet\\

\subsection{En nobelpris til norsk nevrovitenskap}

hippocampus og cortex entorrhinicus er to hjerneceller som hjelper hjernen vår å danne en GPS over omgivelsene våre. Det har blitt funnet at disse hjernecellene er fraværende i Alzheimers ofre. Forskerne som fikk denne nobel prisen er

\begin{itemize}
    \item \textbf{\textit{John O Keefe }}
    \item \textbf{\textit{May Britt}}
    \item \textbf{\textit{Edvard Moser}}
\end{itemize}




\section{Hva er materialisme og dualisme?}

Man skulle tro at nevrovitenskapen i seg selv er et argument som viser at dualisme bare er tull. Dette er ikke sant. Det er fortsatt filosofer og nevroforskere som tar argument for dualisme på alvor. For å forstå dette, tar vi for oss noen definisjoner\\


Materialisme går ut på at alle følelser og sinnet i seg selv, ikke er noe annet en fysiske tilstander og komposisjoner av hjerneceller og nervesystemer i kroppen vår: Den støtter nevrovitenskap og at følelser kan studeres av naturvitenskaper som fysikk, kjemi og biologi.\\

\begin{description}
    \item[Egenskapsmaterialisme] Alle sinnstilstander er identiske med materielle tilstander 
\end{description}

Motsatt tenker dualisten at noen sinnstilstander er grunneleggende forskjellige fra alle sinnstilstander

\begin{description}
    \item[Egenskapsdualisme] Noen sinnstilstander er grunnleggende forskjellige fra materielle tilstander 
\end{description}


Disse to definisjonene tar for seg egenskaper som kroppen vår har. Altså at enkelte sinnstilstander er materielle tilstander. Elisabeth og Descartes diskuterer hvorvidt hver og en av oss er en materiell ting. Debatten deres handler ikke om hva sinnstilstander er, men om hvilken type ting vi er. Filosofer kaller ofte en ting for "substans"


\begin{description}
    \item[Substansmaterialisme] Alle ting er materielle
\end{description}

\begin{description}
    \item[Substansdualisme] Sinnsting (oss selv inkludert) er ikke materielle ting 
\end{description}


\section{Skillet mellom korrelasjon og identitet}

ta en lysbryter og et lys. Stillingen til lysbryteren og hvorvidt lyset er på, henger direkte sammen, men stillingen i lysbryteren er ikke identisk med tilstanden til lyspæren.\\

Dersom materielle tilstander og sinnstilstander henger sammen slik, er det ikke egenskapsmaterialisme; sinnstilstanden kan jo være fundamentalt forskjellige fra de materielle tilstandene og fremdeles være forårsaket av dem. \\

Det samme gjelder for substansmaterialisme; Det faktum at hjernen min er direkte knyttet til sinnet, viser ikke at substansmaterialisme er sant. Kanskje sinnet bare samhandler med hjernen min.\\

Dualisme er mer kompatibelt med nær samhandling mellom hjerne og sinn.\\

Det å si at $ a=b $, betyr det at a er identisk med b. ("sinnstilstanden er identisk med hjernetilstanden") Men dette er ikke riktig måte å tenke på. Det er bare en ting vi spør om er sant eller ikke. Vi kan ikke si at to personer er identiske, det vil si at de har akkurat lik kompisjon av celler som bygger opp akkurat samme menneske. Hvis sinnet og hjernen er indetiske, er det ikke mulig å adskille dem. \\

Phineas Gage sitt tilfelle viser bare at sinnstilstander og hjernetilstander er nært forbundet (som lysbryter og lyspære). \textbf{\textit{Materialisme krever at sinn og materie ikke på noen måte kan skilles fra hverandre}}.\\

Det mest kjente argumentet for dualisme finner vi hos Descartes.

\textbf{\textit{Hvis vi kan tenke oss at sinner eksisterer uavhengig av kroppen, da kan ikke sinnet være identisk med materien}}. Materien er ustrakt i rommet, mens sinnet er tenkende og uten utstrekning. SIden bevissthetens natur ikke kan reduseres til utstrekning eller mekanikk, kan ikke alt forklares som materie alene



\section{Argumentet mot dualisme}

Elisabeth bruker "Reductio ad absurdum" altså vi antar et dualistisk syn, og utleder en helt abdsurd påstand fra den. Dette skal bevise at dualisme ikke kan være sant.\\

Elisabeths spesifikke innvending kalles \textit{Innvendingen ut ifra mental årsakssammenheng}. Ta for eksempel at jeg tenker jeg skal lage en kopp kaffe. Da er det en tanke og en sinnstilstand som får meg opp å lage en kopp kaffe. Men om sinnet selv ikke er en materiell ting (og sinnstilstander ikke er materielle tilstander), hvordan kunne de forårsake endringer i den materielle verden? Det samme gjelder å ta på noe varmt. Det at det ikke er noe materialistisk som får oss til å ta vekk hånden er også absurd. Dette kalles for \textbf{\textit{den kausale lukningen}} mot den materielle verden. Det sentrale problemet Elisabeth peker på er at dersom sinnet ikke er en materiell ting, skulle den ikke forårsake endringer i den materielle verden.


\section{studiespørsmål}

\begin{enumerate}
    \item Hva er Descartes argument for dualisme, hvis det finnes, i den andre meditasjonen?
    \item Utgjør Descartes argument egentlig substansdualisme (at mentale ting ikke er materielle ting), eller utgjør det bare egenskapsdualisme (at tenke-tilstanden ikke er en materiell ting)?
    \item Hvordan argumenterer Elisabeth for at Descartes dualisme ikke kan forklare mental årsakssammenheng?
    \item Synes du at Descartes svar på Elisabeths agument er overbevisende?
\end{enumerate}

\subsection{Dette bør du kunne}

\begin{enumerate}
    \item Definere henholdsvis kropp-sinn dualisme og kropp-sinn materialisme og skille mellom forskjellige versjoner av dem
    \item Reflektere kritisk over Descartes argument for kropp-sinn dualismen
    \item Forstå og vøre i stand til å reflektere kritisk over Elisabeths innvending mot kropp-sinn dualismen
\end{enumerate}





\end{document}