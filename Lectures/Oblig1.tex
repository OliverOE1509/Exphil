\documentclass[11pt, a4paper]{article}

% ===== ESSENTIAL PACKAGES FOR PHILOSOPHY =====
\usepackage[english]{babel} % For hyphenation and linguistic rules
\usepackage{geometry} % For comfortable margins for reading
\geometry{a4paper, margin=1.2in} % Slightly larger margins for notes
\usepackage{setspace} % For line spacing
\onehalfspacing % 1.5 line spacing for readability

% ===== PACKAGES FOR LOGIC & ARGUMENT STRUCTURE =====
\usepackage{logicproof} % For formal logic proofs (optional)
\usepackage{enumitem} % For customizing lists

% ===== CUSTOM ENVIRONMENTS FOR PHILOSOPHICAL ARGUMENTS =====
\newenvironment{argument}{\begin{quote}\itshape}{\end{quote}} % For quoting arguments
\newenvironment{objection}{\begin{quote}\itshape\textbf{Objection: }}{\end{quote}} % For objections
\newenvironment{reply}{\begin{quote}\itshape\textbf{Reply: }}{\end{quote}} % For replies

% ===== DOCUMENT BEGIN =====
\title{Kap 6 Er følelser et hinder for kunnskap?}
\author{Oliver Ekeberg}
\date{}

\begin{document}
\maketitle

Word count:

\tableofcontents

% No abstract is typical for shorter philosophy essays

\section{Oppbygning}

\textbf{\textit{Total oppbygning: 100}}

\begin{enumerate}
    \item Begynner med innledning (15)
    \item Hoveddel (75)
    \item Konklusjon (10)
\end{enumerate}

\section{Abstrakt}

I denne teksten skal jeg analysere argumentene Descartes gjør i avsnitt 11 og 12 i andre meditasjon. 
Siden det å analysere et argument, betyr å tolke begrunnelser og premissene som er lagt frem, skal jeg gjøre klart standpunktene om hva Descartes mener om det å tenke selv. Det 
som har vært, vil ikke alltid være. Dette har filosofisk interesse fordi vi er interessert i hvordan vi kan stole på sansene våre til å vite at vi er sanne. Descartes mente selv at det er et ondt vesen som er manipulativ, ondsinnet og en forfører som overbeviser samvittigheten vår om at vi er i live.
Descartes sin løsning var å tvile alt, og bygge opp alt han vet fra bunnen av. Dette gjorde han med rasjonalitet og matematisk logikk. Iterativt, så vil man få en pyramide som består av kunnskap du vet er sant, fordi du har bevist at forføreren ikke har plantet det i samvittigheten din.





\section{Hoveddel}

I denne delen skal jeg redegjøre for alle argumenter Descartes lager i teksten sin. Dette skal gjøres trinnvis.  
Jeg skal tolke hva Descartes faktisk mener med at legemet som er voksen fra bikuben, ikke er 
sødme, blomsterduft, eller lyden den lager. Men at voksen er et legemet som for en stund siden var et fast objekt 
som man kunne banke på en stein, og som nå er blitt \textbf{\textit{omformet}} til et annet legeme.


Descartes åpner opp ved å snakke om de objekter vi kan alle være enige om at eksisterer. Vi kan ta på de, føle. de, lukte de, se de osv. Han tar et voksstykke fra en bikube som et eksempel. I et øyeblikk har den en fast form, og vi ser det ved å lage et bildeform av voksstykket. Descartes bringer voksstykket så over ilden hvor det smelter. Han observerer at lukten, fargen og formen forsvinner. Han argumenterer så for at voksstykket er det samme, selv etter at det har smeltet. Han opplyser først om at ingen tviler på det. 

\begin{objection}
    Argumentet om at ingen tviler på det er ikke et veldig sterkt argument. Men man kan argumentere for at det er rasjonelt å anta at ingen tviler på det.
\end{objection}


Descartes spør så hva er det som gjør at det er samme objekt? De attributene som vi beskrev voksstykket er ikke lenger tilstedeværende. Så hva er det vi forestiller oss?


Descartes argumenterer videre for at voksstykke er bare noe vi billedlig forestiller oss, altså noe som er utstrakt, bøyelig og bevegelig. Dette billedlige objektet kan ha uendelige mange former.

\section{Konklusjon}

Oppsummer hvordan vokset viser hvordan intellektet er viktigere enn sansene i erkjennelsen



\end{document}