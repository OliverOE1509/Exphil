\documentclass[11pt, a4paper]{article}

% ===== ESSENTIAL PACKAGES =====
\usepackage{geometry}
\geometry{a4paper, margin=1.2in}
\usepackage{setspace}
\onehalfspacing

\usepackage{catchfile}

% ===== WORD COUNT COMMAND =====
\newcommand{\wordcount}{%
    \ifnum\pdf@shellescape=1%
        \immediate\write18{texcount -1 -sum Oblig1.tex > wordcount.txt}%
        \CatchFileDef{\words}{wordcount.txt}{}%
        \words%
    \else%
        [Enable -shell-escape]%
    \fi%
}

% ===== CUSTOM ENVIRONMENTS FOR PHILOSOPHICAL ARGUMENTS =====
\newenvironment{argument}{\begin{quote}\itshape\textbf{Argument: }}{\end{quote}}
\newenvironment{objection}{\begin{quote}\itshape\textbf{Innvending: }}{\end{quote}}
\newenvironment{reply}{\begin{quote}\itshape\textbf{Reply: }}{\end{quote}}

% ===== DOCUMENT BEGIN =====
\title{Obligatorisk oppgave nr. 1}
\author{Oliver Ekeberg}
\date{}

\begin{document}
\maketitle

Word count: 607


\section{Abstrakt}



I denne oppgaven skal jeg analysere Descartes' argumentasjon i \textit{Meditasjoner over den første filosofi}, spesielt med fokus på hans bruk av voksstykket som et eksempel for å illustrere forholdet mellom sansene og intellektet i erkjennelsen av virkeligheten. Jeg vil undersøke hvordan Descartes argumenterer for at vår forståelse av objekter ikke er avhengig av sansene. Han mener at for å tro på at noe eksisterer, må det kunne gripes med ånden vår, eller intellektet som jeg vil referere det til. Videre trodde Descartes at det er en ond demon som er ute etter å forføre og bedra oss, og han mente at man ikke kan stole på sansene våre av den grunn.


\section{Hoveddel}


Descartes åpner opp ved å snakke om de objekter vi kan alle være enige om at eksisterer. Han tar et voksstykke som et eksempel. I et øyeblikk har den en fast form, og spesifikke attributter. Descartes tar så voksstykket over ilden hvor det smelter. Han observerer at lukten, fargen og formen forsvinner. Descartes spør så hva er det som gjør at voksstykket er samme legeme? De attributene som vi beskrev voksstykket med, er ikke lenger tilstedeværende. Så hva er det vi forestiller oss? 


Descartes ber oss om å fjerne alt som tilhører vokset, for å se hva som blir igjen. Han sier at det som er igjen, er noe utstrakt, bevegelig og bøyelig. Dette kan ha en uendelig mengde med former, noe som mennesker ikke er i stand til å forestille seg. Han argumenterer så for følgende:


\begin{argument}
    Hva vokset er, er jeg ikke i stand til å forestille meg: det kan jeg bare gripe med ånden. "\textit{Sola mente percipere}" (Cappelen, Torsen, Watzl s. 41). 
\end{argument}




Jeg bruker Leontiev (2020) for en oversettelse av \textit{sola mente percipere} "\textit{By the mind alone}" (Leontiev, 2020)


Voksstykket er et eksempel som klart skiller sanser og intellekt. Dette er fordi idet han bringer vokset over ilden, forsvinner attributtene, men det er fortsatt samme voksstykket. Intellektet vil se det underliggende og utstrakte som vil gi sikker kunnskap om at voksstykket fortsatt er samme voksstykket. Det samme voksstykket som bare kan gripes av ånden, er det samme stykket han antok fra starten av.

Descartes argumenterer så følgende

\begin{argument}
    Erkjennelsen av vokset er ene og alene et åndens innsyn "\textit{solius mentis inspectio}" (Cappelen, Torsen, Watzl s. 41). 
\end{argument}


Jeg bruker Leontiev (2020) for en oversettelse av \textit{solius mentis inspectio} "\textit{But of purely mental scrutiny}" (Leontiev, 2020)

Dette argumentet kan tolkes som at det er bare ånden, eller intellektet til et menneske som kan erkjenne at voksstykket eksisterer. Det dette argumentet sier er at erkjennelse av vokset er utelukkende et intellektens syn. Ta en kopp kaffe, som er varm og luter kaffe. Etter noen timer, er det en kald og luktløs kopp med kaffe. Hvis du bare har sansene dine til å beslutte, vil den si at det er en annen kopp, samtidig som at intellektet vet at det er samme kopp med kaffe. En kombinasjon av disse to argumentene, vil gi oss sikker kunnskap vi kan stole på

\section{Konklusjon}

For å oppsummere, kan vi ikke stole på at sansene våre sier at det er samme voksstykket, og det er umulig å skjønne at voksstykket er det samme etter brenningen, uten hjelp av intellektet vårt. Den andre meditasjon gir en støttemur for hvordan vi kan finne oss frem til sikker kunnskap. 




\section{Litteraturliste}

\begin{itemize}
    \item Cappelen, H., Torsen, I., Watzl, S. (2021). Vite, være, gjøre: Exphil (s. 40-41). Gyldendal
    \item Leontiev, D. (2020). Meditations on first philosophy: The Wax Argument and the Cartesian Mind. PhilArchive. https://philarchive.org/archive/LEOTMP-4 (s. 6)
    \item Leontiev, D. (2020). Meditations on first philosophy: The Wax Argument and the Cartesian Mind. PhilArchive. https://philarchive.org/archive/LEOTMP-4 (s. 6)
\end{itemize}


\end{document}