\documentclass[11pt, a4paper]{article}

% ===== ESSENTIAL PACKAGES FOR PHILOSOPHY =====
\usepackage[english]{babel} % For hyphenation and linguistic rules
\usepackage{geometry} % For comfortable margins for reading
\geometry{a4paper, margin=1.2in} % Slightly larger margins for notes
\usepackage{setspace} % For line spacing
\onehalfspacing % 1.5 line spacing for readability


% ===== CUSTOM ENVIRONMENTS FOR PHILOSOPHICAL ARGUMENTS =====
\newenvironment{argument}{\begin{quote}\itshape\textbf{Argument: }}{\end{quote}} % For quoting arguments
\newenvironment{objection}{\begin{quote}\itshape\textbf{Innvending: }}{\end{quote}} % For objections
\newenvironment{reply}{\begin{quote}\itshape\textbf{Reply: }}{\end{quote}} % For replies

% ===== DOCUMENT BEGIN =====
\title{Kap 6 Er følelser et hinder for kunnskap?}
\author{Oliver Ekeberg}
\date{}

\begin{document}
\maketitle

Word count:

\tableofcontents

% No abstract is typical for shorter philosophy essays

\section{Oppbygning}

\textbf{\textit{Total oppbygning: 100}}

\begin{enumerate}
    \item Begynner med innledning (15)
    \item Hoveddel (75)
    \item Konklusjon (10)
\end{enumerate}

\section{Abstrakt}



I denne oppgaven skal jeg analysere Descartes' argumentasjon i \textit{Meditasjoner over den første filosofi}, spesielt med fokus på hans bruk av voksstykket som et eksempel for å illustrere forholdet mellom sansene og intellektet i erkjennelsen av virkeligheten. Jeg vil undersøke hvordan Descartes argumenterer for at vår forståelse av objekter ikke er avhengig av sansene. Descartes var klar på at sanser kan bedra oss, og at en bedragersk og ondskapsfull ånd vil lure oss. Derfor vil ikke sansene våre 


\section{Hoveddel}

I denne delen skal jeg redegjøre for alle argumenter Descartes lager i teksten sin. Dette skal gjøres trinnvis.  
Jeg skal tolke hva Descartes faktisk mener med at legemet som er voksen fra bikuben, ikke er 
sødme, blomsterduft, eller lyden den lager. Men at voksen er et legemet som for en stund siden var et fast objekt 
som man kunne banke på en stein, og som nå er blitt \textbf{\textit{omformet}} til et annet legeme.


Descartes åpner opp ved å snakke om de objekter vi kan alle være enige om at eksisterer. Han tar et voksstykke som et eksempel. I et øyeblikk har den en fast form, og spesifikke attributter. Descartes tar så voksstykket over ilden hvor det smelter. Han observerer at lukten, fargen og formen forsvinner. Descartes spør så hva er det som gjør at voksstykket er samme legeme? De attributene som vi beskrev voksstykket er ikke lenger tilstedeværende. Så hva er det vi forestiller oss? 


Descartes ber oss om å fjerne alt som tilhører vokset, for å se hva som blir igjen. Han sier at det som er igjen, er noe utstrakt, bevegelig og bøyelig. Dette kan ha en uendelig mengde med former, noe som mennesker ikke er i stand til å forestille seg. Han argumenterer så for følgende:


\begin{argument}
    Hva vokset er, er jeg ikke i stand til å forestille meg: det kan jeg bare gripe med ånden \textit{Sola mente percipere}. 
\end{argument}




Descartes er klar på at sansene våre kan bedra oss. Det som har konstruert vår samvittighet kan vær falskt, og at den ene måten å komme seg rundt, er bygge opp sitt intellekt fra bunnen av. Han trodde at et ondskapsfullt geni, kunne bedra oss til å tro ting som ikke er sanne. Blant annet at dette voksstykket er det samme som før. Måten vi kan overvinne denne tvilen, er å bruke det faktum at det er bare ånden som har makten til å forme det til hva den vil.



\begin{argument}
    Erkjennelsen av vokset er ene og alene et åndens innsyn (\textit{solius mentis inspectio}). 
\end{argument}


Denne anerkjennelsen, kan enten være uklar og ufullstendig, altså levende bevis på at ånden har forført oss, eller så kan vi ha en klar og distinkt forståelse av det vi erkjenner. Som åpenbart i dette tilfellet er det samme voksstykket som vi holdte før det smeltet

\section{Konklusjon}

Oppsummer hvordan vokset viser hvordan intellektet er viktigere enn sansene i erkjennelsen



\end{document}