\documentclass[11pt]{article}
\usepackage[a4paper,margin=1.8cm]{geometry}
\usepackage{amsmath, amssymb, amsthm}  % Essential math packages
\usepackage{graphicx}                   % For figures
\usepackage{hyperref}                   % Clickable links
\usepackage{parskip} % cleaner spacing
\usepackage{enumitem}


\newtheorem{definition}{Definisjon}

\setlength{\parskip}{1em}
\setlength{\parindent}{0pt}

\title{Kap 1: Tenk Selv!}
\author{Oliver Ekeberg}
\date{\today}

\begin{document}
\maketitle

\section{Hvorfor tenke selv?}

Det å tenke selv er blitt viktigere enn noen gang på grunn av tilgangen på informasjon.

\vspace{1em}
Eksempel: Hvis jeg googler et svar om hva som er median inntekt i Norge i 2024, så har jeg et intellektuelt ansvar om å stole på den autoriteten som har gitt meg det svaret. 

\vspace{1em}
\begin{itemize}
    \item a. hvorfor er det viktig og verdifullt å tenke selv?
    \item b. hvorfor tenker man selv - har vi metoder som gjør oss til bedre tenkere?
    \item c. hvordan kan et samfunn gjøre det mulig for borgere å tenke selv?
\end{itemize}


\vspace{1em}
svar på a):

Det er viktig for oss å forstå verden vi lever i. For å forstå den må vi ha kunnskap om den, og beste måten å få kunnskap på er å tenke selv

\vspace{1em}
Hvorfor kan vi stole på at vi vet hvordan man finner svaret? Vi er tross alt kropper av fett og vann i et uendelig univers. 



\section{Tenkingens fiender: slapphet og feighet}

Hva hindrer oss i å tenke selv? Immanuel Kant mener at det å tenke selv er hva som gjør oss til et "opplyst menneske"

To faktorer hindrer oss i å tenke selv:
\begin{enumerate}
    \item \textbf{Slapphet:} Det å tenke selv krever energi. Det er lett å velge å stole på autoriteter som legen din, generalen din eller datamaskinen din
    \item \textbf{Feighet:} Det å akspetere autoriteter kan være hva som er nødvendig for å beholde livet ditt. Det er også risikabelt å tenke selv - fordi det er lett å ta feil
\end{enumerate}

\vspace{1em}
Så for å tenke selv må vi overkomme slapphet og feighet. Vi må også bo i et samfunn hvor det er lov å tenke selv, og med tilgang på informasjon og kunnskap

\section{Et skille mellom offentlig og privat tenkning?}

Kant mente også kontroversielt at:

\vspace{1em}
Hvis man jobber for en stat, selskap, organisasjon eller kirke; så skal man adlyde hva de sier, altså ikke uttrykke uenighet.




\begin{definition}
    Opplysningstiden: En periode fra slutten av 1600-tallet til begynnelsen av 1800-tallet. europeisk indellektuell historie, avviste tradisjonelle autoriteter (særlig kirken). Anså menneskelig fornuft og empirisk data for å oppnå kunnskap og fremheve nye idealer
\end{definition}



\section{Hvordan kan vi tenke selv?}

Rene Descartes var interessert i å finne ut "hvordan man tenker selv". Han mener at man må bygge opp sin egen mening som en pyramide. De viktigste steinene i en pyramide er helt nederst. Hvis de ikke er sikre, vil hele pyramiden rase sammen, og argumentet ditt faller sammen.
\vspace{1em}

Descartes mener at for å tenke selv må du ha et fundament. Om vi presenterer et argument med premiss som bare er fortalt av  et annet menneske, kan stole på det premisset? I utgangs
punktet; Nei!
\vspace{1em}

Et viktig premiss i Descartes sitt argument er at kunnskapen vår er hierarkisk: Hvis de fundamentale antagelsene er usikre, faller alt som bygger på dem, sammen
\vspace{1em}
For eksempel: Hvordan kan vi stole på sansene våre? Vi vet at vi kan hallusinere og drømme. Hvis vi drømmer at vi spiser kake, så gjør vi det ikke. 

\vspace{1em}
Jeg har en kritikk til dette. Jeg mener at sansene våre er det som gir oss retning i meninger og perspektivene våre. Sier Descartes at vi ikke skal stole på intuisjonen vår? Intuisjonen vår er er en viktig del av vår samvittighet. Og det som gir oss følelser, medlidenhet, sinne og tristhet, kommer fra samvittigheten. Det er hva som gjør oss til et menneske.


\section{Tvil og skeptisisme}

Descartes fortalte seg selv at onde ånder  prøvde å forføre og mislede han. Idag er det mer relevant å tenke at man bor i en Data Simulasjon.
\vspace{1em}
Elon Musk, og Nick Bostrøm har sagt det samme. De argumenterer for at teknologisk fremskritt vil gjøre sivilisasjoner i stand til å generere simulasjoner så detaljerte at de kan gi ethvert menneske en samvittighet, og dermed kunne kontrollere  den samvittigheten.

\vspace{1em}
Resultatet er en ekstrem form for skeptisisme


\textbf{skeptisisme: } Er filosofiske argument som skal vise oss at  vi ikke vet, det vi tror vi vet.

Descartes er ute etter å finne kriterier for sikker kunnskap, slik at tvil er en metode for å skille mellom ulike kunnskapstyper



\section{Studiespørsmål:}

\begin{enumerate}
    \item \textbf{Hva er opplysningstidens motto og hvordan kan man ifølge Kant leve opp til dette?:} Opplysningstidens motto var oppnåelse av kunnskap på bakgrunn av å tenke selv, og empiriske data. Det var en "rejection" av tidligere autoriteter (kirken), og fremheving av nye idealer som demokrati. Ifølge Kant, kan man leve opp til dette mottoet ved å...
    \item \textbf{Hva er forholdet mellom fri tanke og fri handling ifølge Kant?:} Kant mente at fri tanke er en rettighet som gir deg evnen til å tenke selv. Det vil si at istedenfor å stole blindt på autoriteter som leger, diktatorer eller datamaskiner, skal du unngå feighet og slapphet for å tenke selv. Når Kant snakker om fri handling, mener han at du som jobber for en organisasjon som en kirke, et selskap, eller stat. Da har du som plikt å ikke kritisere og argumentere fritt mot sin arbeidsgiver. I en akademisk kontekst, er det akseptabelt å uttrykke uenighet med selskapet ditt. Men det er ingen plass for varslere i Kants filosofi
    \item \textbf{Hvorfor mener Descartes at vi ikke kan stole fullstendig på våre sanseinntrykk når vi forsøker å finne et sikkert fundament for kunnskap?:} Descartes mener at sansene våre kan bli lurt og forført av onde ånder. Dette vil si at vi ikke kan stole på sansene våre. Mennesket kan drømme og hallusinere for eksempel. Da blir vi forført av våre egne sanser. Hvis vi drømmer at vi spiser kake, gjør ikke egentlig det. En moderne anvendelse av dette argumentet er at teknologiske samfunn blir så avansert og så utviklet, slik at simulasjoner blir så nøyaktige at de kan simulere sanser  og samvittighetene til mennesker. Det å finne et sikkert fundament for kunnskap vil si at vi må bygge opp kunnskapen vår som en pyramide. I en pyramide så er de nederste steinene de viktigste, som bærer hele strukturen og vekten til pyramiden. Hvis de svikter, svikter hele pyramiden. Og hvis vi ikke kan stole på sansene våre, er det ingen god ide å bruke sanser som fundament i oppnåelsen av kunnskap.
    \item \textbf{Hvilke tankeeksperiment hos Descartes kan sammenlignes med tankeeksperiment som går ut på at vi lever i en datasimulering?:} Dette sa jeg allerede i forrige oppgave
    \item \textbf{Hva er det eneste vi kan vite helt sikkert, ifølge Descartes?:} Descartes mener at vi kan helt sikkert vite at vi ikke kan stole på alt vi sanser. Alt vi får som kunnskap er en funksjon av sansene våre. Derfor finnes det tvil i nesten alt man skal tro på.
\end{enumerate}

Dette bør du kunne:
\begin{enumerate}
    \item \textbf{Forklare hva Kant mener med opplysning:} Det Kant mente med opplysning er evnen til å tenke selv. Han mente at det som gjør et menneske opplyst, er evnen til å tenke selv. Det Det går to faktorer under evnen til å tenke selv: 1) Slapphet, og 2) Feighet.
    \item \textbf{Være i stand til å forstå og reflektere over hvorfor og hvordan man bør tenke selv:} Hvorfor bør man tenke selv? En kan si at det er viktig fordi det vi er en del av verdenen vi lever i. Hvorfor skal akkurat jeg tenke selv? Kan et dyr tenke selv? Vi har som regel to alternativer: 1) Tenke selv over argument, og 2) Overføre til en autoritet. Da må vi kunne stole på at den autoriteten gir oss riktig svar. Hvis vi får feil svar blir vi misinformert
    \item \textbf{reflektere rundt hvordan vi tilegner oss kunnskap, og hvilke kilder vi kan stole på når vi tilegner oss kunnskap:} Når vi tilegner oss kunnskap, mener Descartes for eksempel at vi skal bygge opp et fundament av det vi vet er helt sikkert. For så å bygge oppover fra denne steinen. Derfor er det så viktig at dette fundamentet er sikkert. Hvis det er usikkert, risikerer man at hele pyramiden raser sammen
    \item \textbf{Forstå og forklare Descartes skeptiske argument:} Descates skeptiske argument betyr at det å tenke og tvile, er det som er fundamentet for all kunnskap. Skepsis er en form for tvil. Og det å tvile, betyr at du bygger et fundament som er sikkert. Dette kommer av å tenke. 
\end{enumerate}

\end{document}

