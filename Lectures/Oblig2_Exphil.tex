\documentclass[12pt, a4paper]{article}

% ===== ESSENTIAL PACKAGES =====
\usepackage{geometry}
\geometry{a4paper, margin=1in}
\usepackage{setspace}
\onehalfspacing
\usepackage{hyperref} % legg dette i preamble

\usepackage{catchfile}

% ===== WORD COUNT COMMAND =====
\newcommand{\wordcount}{%
    \ifnum\pdf@shellescape=1%
        \immediate\write18{texcount -1 -sum Oblig1.tex > wordcount.txt}%
        \CatchFileDef{\words}{wordcount.txt}{}%
        \words%
    \else%
        [Enable -shell-escape]%
    \fi%
}

% ===== CUSTOM ENVIRONMENTS FOR PHILOSOPHICAL ARGUMENTS =====
\newenvironment{argument}{\begin{quote}\itshape\textbf{Argument: }}{\end{quote}}
\newenvironment{objection}{\begin{quote}\itshape\textbf{Innvending: }}{\end{quote}}
\newenvironment{reply}{\begin{quote}\itshape\textbf{Reply: }}{\end{quote}}
\newenvironment{oversettelse}{\begin{quote}\itshape\textbf{Oversettelse: }}{\end{quote}}

% ===== DOCUMENT BEGIN =====
\title{Exphil03 Obligatorisk oppgave nr. 2}
\author{Oliver Ekeberg}
\date{\today}



\begin{document}
\maketitle

\tableofcontents

\begin{argument}
    ``menneskesjelen, som kun er en tenkende substans, kan styre kroppens ånder [esprits] og forårsake frivillige handlinger''
\end{argument}

\section{Innledning}

Descartes er en filosof fra 1600-tallet som oppfattes som grunnleggeren av substansdualisme, nemlig teorien om at sinnet og kroppen vår består av to fundamentalt forskjellige substanser; Den tenkende sjelen (res cogito) og den utstrakte kroppen (res extensa). Descartes levde i en tid der troen på et mekanistisk univers stadig ble mer populær. Nemlig at den naturen vår kan forklares ved hjelp av fysiske lover og matematiske betingelser. Descartes hevder likevel at den utstrakte verdenen kan påvirkes direkte fra sjelen vår, som ifølge han selv er ikke utstrakt. Elisabeth utfordrer denne ideen ved å utlede at det er absurd. For Elisabeth er en betingelse for utstrakt bevegelse, at det må virkes på av noe annet utstrakt. Elisabeth sier altså ikke at det ikke finnes noe mentalt. Hun er altså ikke en materialist, men hun sier at det ikke går ann at sjelen er en adskilt substans og at det samtidig kan påvirke noe utstrakt.

Problemstillingen i denne teksten er derfor å utforske hvordan Descartes kan forsvare sitt argument for substansdualisme, og hvordan Elisabeths argument viser svakhetene i argumentet hans.
\section{Bakgrunn}

Dualisme innebærer eksistens av to substanser, det fysiske og det mentale. Det fysiske skulle være styrt av mekanistiske lover, som i bunn og grunn betyr at hele universet er en maskin, som kjører uten en grunn eller en mening (res extensa)

\begin{oversettelse}
    Res extensa: Fra latin til engelsk: ``Extended thing``. Fra Engelsk til Norsk: ``Utstrakt ting``
\end{oversettelse}

Det mentale substanset er ikke-utstrakt, og har ingen påvirkning fra den mekanistiske maskinen

\begin{oversettelse}
    Res Cogitans: Oversatt fra lating til engelsk: "thinking thing". Fra Engelsk til Norsk: "Tenkende ting" 
\end{oversettelse}





\section{Descartes argument}

En viktig antagelse for å godta Descartes argument, er nemlig at sjelen i seg selv eksisterer. Dette er ikke akseptert. Descartes argumenterer for dette med sitt argument

\begin{argument}
    \textbf{\textit{Cogito ergo sum}} - Jeg kan tvile på kroppen min og sanser den forteller meg, men ikke at jeg tenker. Dermed må jeg eksistere som en tenkende substans, uavhengig av kroppen.
\end{argument}

Dette er et grunnlag Descartes lager for at sjelen i seg selv er en substans. Han har så konkludert med at mennesket forklares av to forskjellige substanser

\begin{enumerate}
    \item \textbf{\textit{Sjelen:}} Kjennetegnes av tenkning og ikke-utstrakt
    \item \textbf{\textit{Kroppen:}} Kjennetegnes av utstrekning, og at det er underlagt mekaniske lover.
\end{enumerate}

Det er dette Descartes refererer til som substansdualisme. Han poengterer videre at sjelen kan påvirke kroppen vår. Dette er rimelig å anta, siden våre handlinger er styrt av for eksempel motivasjon og vilje. Descartes argumenterer for at denne overgangen skjer i \textit{pinealkjertelen}. Herfra styrer sjelen \textit{esprits} - "dyreånder" som beveger seg i nervene våre og setter kroppen i bevegelse.

\begin{argument}
    \textbf{\textit{Esprits}} - dyreånder som reiser fra pinealkjertelen og til kroppen vår, som lager bevegelse
\end{argument}

Kjedereaksjonen fra sjel til kropp kan ses på slik fra Descartes sitt argument.

minne $\rightarrow$ følelse (tristhet) $\rightarrow$ esprits $\rightarrow$ neuroner $\rightarrow$ tårer


En sterk antagelse Descartes tar, er at sinnet vårt er ikke ustrakt og immateriell.

\section{Elisabeths kritikk}

Elisabeth sin hovedkritikk, er at for at noe fysisk skal kunne påvirkes, så må det være noe som virker på den. 
\begin{argument}
    \textbf{\textit{Reductio ad absurdum}} - En motbevisning fra å anta at noe er sant, og at det leder til en absurd konklusjon
\end{argument}

Elisabeth argumenterer for at det er absurd at noe ikke-utstrakt kan påvirke hvordan vi føler oss. 

Kravet som Elisabeth stiller til bevegelse, er at den bevegende tingen dyttes på en spesiell måte, eller ved de spesielle egenskapene, og at det som virker på objektet er utstrakt. Siden Descartes har definert sjelen som noe immaterielt og tenkende, er det uforenlig at sjelen skal kunne ha noen påvirkning på den mekanistiske verden. 

\section{Analyse/Drøfting}


\textbf{\textit{Det kan høres ut som at Elisabeth er en egenskapsmaterialist. Altså at alle sinnstilstander er identiske med materielle tilstander (exphil boka s.211)}}

Nå som både Descartes og Elisabeth har presentert sine argument, kan jeg drøfte uenigheten deres. Elisabeth benekter som sagt ikke dualisme i seg selv, men hun kan ikke anta at sjelen er ikke-utstrakt. Hun legger til grunn for en ny type dualisme;  egenskapsdualisme

\section{Konklusjon}

\textbf{\textit{Oppsummer hva Descartes mente og hvorfor han møtte problemer.\\
Fremhev Elisabeths betydning: hun peker på mind–body problemet som fortsatt diskuteres i moderne filosofi og kognitiv vitenskap.\\
Avslutt med å vise at diskusjonen mellom dem illustrerer et grunnleggende filosofisk problem: kan det finnes kausal interaksjon mellom to helt forskjellige substanser?}}

\section{Referanser}


\begin{itemize}
    \item Substans av Descartes: \url{https://plato.stanford.edu/entries/substance/}.
    \item PinealKjertel Descartes: \url{https://www.britannica.com/science/death/Descartes-the-pineal-soul-and-brain-stem-death}
    \item Descartes til Elisabeth 21. mai 1643 
\end{itemize}



\end{document}