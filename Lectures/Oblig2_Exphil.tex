\documentclass[12pt, a4paper]{article}

% ===== ESSENTIAL PACKAGES =====
\usepackage{geometry}
\geometry{a4paper, margin=1in}
\usepackage{setspace}
\onehalfspacing
\usepackage{hyperref} % legg dette i preamble
\usepackage{amsthm}
\newtheorem{definition}{Definisjon}
\usepackage{catchfile}

% ===== WORD COUNT COMMAND =====
\newcommand{\wordcount}{%
    \ifnum\pdf@shellescape=1%
        \immediate\write18{texcount -1 -sum Oblig1.tex > wordcount.txt}%
        \CatchFileDef{\words}{wordcount.txt}{}%
        \words%
    \else%
        [Enable -shell-escape]%
    \fi%
}

% ===== CUSTOM ENVIRONMENTS FOR PHILOSOPHICAL ARGUMENTS =====
\newenvironment{argument}{\begin{quote}\itshape\textbf{Argument: }}{\end{quote}}
\newenvironment{objection}{\begin{quote}\itshape\textbf{Innvending: }}{\end{quote}}
\newenvironment{reply}{\begin{quote}\itshape\textbf{Reply: }}{\end{quote}}
\newenvironment{oversettelse}{\begin{quote}\itshape\textbf{Oversettelse: }}{\end{quote}}

% ===== DOCUMENT BEGIN =====
\title{Exphil03 Obligatorisk oppgave nr. 2}
\author{Oliver Ekeberg}
\date{\today}



\begin{document}
\maketitle

\tableofcontents


\section{Innledning}

Descartes (1596-1650) oppfattes som grunnleggeren av substansdualisme, nemlig teorien om at sinnet og kroppen vår består av to fundamentalt distinkte og separable substanser, den tenkende sjelen (res cogitans) og den utstrakte kroppen (res extensa). Descartes levde i en tid der troen på et mekanistisk univers stadig ble mer populær, dvs. at naturen kan forklares ved hjelp av fysiske lover og matematiske betingelser (Cappelen, Torsen, Watzl, 2021, kap 9.). Descartes hevder likevel at den utstrakte verdenen kan påvirkes direkte fra sjelen vår, som ifølge han selv er ikke utstrakt. Elisabeth (1596-1662) utfordrer denne ideen ved å vise at det leder til noe absurd. For Elisabeth er en betingelse for utstrakt bevegelse, at det må virkes på av noe annet utstrakt. Elisabeth er ikke en materialist, men hun sier at det ikke går an at sjelen er en adskilt substans og at det samtidig kan påvirke noe utstrakt.

\medskip
Problemstillingen i denne teksten er derfor å utforske hvordan Descartes kan forsvare sitt argument for substansdualisme, og hvordan Elisabeths argument viser svakhetene i argumentet hans.



\section{Descartes argument}

En viktig antagelse for å godta Descartes argument, er nemlig at sjelen i seg selv eksisterer. Dette er fortsatt en omstridt debatt. Descartes argumenterer for dette med sitt argument

\begin{argument}
    \textbf{\textit{Cogito ergo sum}} - Jeg kan tvile på kroppen min og sanser den forteller meg, men ikke at jeg tenker. Dermed må jeg eksistere som en tenkende substans, uavhengig av kroppen.
\end{argument}

Dette er et grunnlag Descartes lager for at sjelen i seg selv er en substans. Han har så konkludert med at mennesket forklares av to forskjellige substanser

\begin{enumerate}
    \item \textbf{\textit{Sjelen:}} Kjennetegnes av tenkning og ikke-utstrakt
    \begin{oversettelse}
    Res extensa (Wikipedia, n.d.): Fra latin til engelsk: ``Extended thing``. Fra Engelsk til Norsk: ``Utstrakt ting``
    \end{oversettelse}
    \item \textbf{\textit{Kroppen:}} Kjennetegnes av utstrekning, og at det er underlagt mekaniske lover.
    \begin{oversettelse}
    Res Cogitans (Merriam-Webster, n.d.): Oversatt fra lating til engelsk: "thinking thing". Fra Engelsk til Norsk: "Tenkende ting" 
    \end{oversettelse}
\end{enumerate}

Det er dette Descartes refererer til som substansdualisme. Argumentasjonen for substansdualisme er at hvis to ting kan forstås klart og distinkt uten hverandre, må de være to ulike substanser. Han poengterer videre at sjelen kan påvirke kroppen vår. Descartes hevder at dette forklarer hvordan sjelen forårsaker frivillig handling, siden våre handlinger er styrt av for eksempel motivasjon og vilje. Problemet er hvordan han kan mene dette. Descartes rettferdiggjør samhandlingen mellom sjel og kropp, ved at det skjer i \textit{pinealkjertelen} (glandula pinealis) (Encyclopaedia Britannica, n.d.). Herfra styrer sjelen \textit{esprits} - "dyreånder" som beveger seg i nervene våre og setter kroppen i bevegelse.

\begin{argument}
    \textbf{\textit{Esprits}} - dyreånder som reiser fra pinealkjertelen og til kroppen vår, som lager bevegelse
\end{argument}



\section{Elisabeths kritikk}

Elisabeth sin hovedkritikk, er at for at noe fysisk skal kunne påvirkes, så må det være noe som virker på den. Elisabeth er ikke materialist, men hun kan ikke godta at en ikke-utstrakt sjel samtidig kan styre kroppens ånder.

\begin{argument}
    \textbf{\textit{Reductio ad absurdum}} - En motbevisning fra å anta at noe er sant, og at det leder til en absurd konklusjon
\end{argument}

Elisabeth argumenterer for at det er absurd at noe ikke-utstrakt kan påvirke hvordan vi føler oss. 

Kravet som Elisabeth stiller til bevegelse, er at den bevegende tingen dyttes på en spesiell måte, eller ved de spesielle egenskapene, og at det som virker på objektet er utstrakt (Cappelen, Torsen, Watzl, 2021, s. 216). Siden Descartes har definert sjelen som noe immaterielt og tenkende, er det uforenlig for Elisabeth at sjelen skal kunne ha noen påvirkning på den mekanistiske verden. 

\section{Analyse/Drøfting}

Elisabeth benekter som sagt ikke dualisme i seg selv, men hun kan ikke anta at sjelen er ikke-utstrakt. Hun viser ved sitt \textit{reductio ad absurdum} argument at dersom sjelen mangler utstrekning, kan det ikke påvirke materie. Elisabeth utvikler ikke en ny terori for dualisme, men hennes innvendning peker i retning av egenskapsdualisme

\begin{quote}
    \textbf{\textit{Egenskapsdualisme}} - Noen sinnstilstander er grunnleggende forskjellige fra materielle tilstander.
\end{quote}

Ideen ble grunnlagt etter Elisabeths tid på 1960-tallet av filosofer David Chalmers (jf. \textit{Store Norske Leksikon, n.d.}), men den løser delvis problemet hun peker på. Et objekt kan ha radikalt ulike egenskaper om man ser på form og farge av det objektet. På samme måte, kan mennesket bare bestå av en substans, men ha fysiske og mentale tilstander. Hva er det som skiller det mentale fra det fysiske? Her kan man tolke egenskapsdualisme som en form for egenskapsmaterialisme. Det er et problem med korrelasjon og identitet (Cappelen, Torsen, Watzl, 2021, s. 211-213). Materialisten sier at sinnstilstander er identisk med en tilstand i hjernen, mens en egenskapsdualist hevder at de bare er korrelerte. For eksempel: En trist følelse kan ikke dekomponeres til elektriske impulser, selv om den er avhengig av den ifølge egenskapsdualisme. Så egenskapsdualisme prøver å bevare egenskapene til bevisstheten, uten å introdusere en ny substans slik Descartes gjorde.

Problemet med dette igjen er at det antar at det ikke er noen sjel. En sjel kan ikke ha utstrekning med mentale egenskaper, fordi sjelen eksisterer uavhengig av kropp, ifølge Descartes

I tillegg til matematiker og filosof, var Descartes en fysiolog som forsket mye på dyrekropper. Her fikk Descartes mest sannsynlig sin logiske forklaring med pinealkjertelen. Men han trodde at dyr ikke hadde noen sjel, og derfor ingen følelser, og kunne derfor bare forske på dyrekropper (vivisection; Encyclopaedia Britannica, n.d.). Det er derfor et svakt argument fra Descartes sin side, men hadde han hatt tilgang til nåtidens nevrologi kunnskap, kunne han kanskje funnet bedre forklaringer på sin teori. Det kan derfor tolkes at eksistens av sjel blir mer et spørsmål om teologi og religion, enn spørsmål om kunnskap.

\section{Konklusjon}

Descartes og Elisabeth hadde begge tro på dualisme, men Elisabeth pekte på en ny retning kalt egenskapsdualisme, fordi hun ikke kunne akseptere Descartes egne substansdualisme. Egenskapsdualismen løser dette problemet delvis ved å tenke på en substans, men med forskjellige egenskaper, som mentalt og fysiske. Egenskapsdualisme kan ikke reduseres til materialisme, fordi materialismen sier at sinnstilstand og elektronisk impuls i hjernen er identiske, mens egenskapsdualismen peker på korrelasjon og avhengigheten en sinnstilstand trenger fra elektroniske impulser. Det er godt mulig at Descartes hadde kommet frem til noe liknende i dag, sett i lys av nevrovitenskapens introduksjon på 1900-tallet.

\section{Referanser}


\begin{itemize}
    \item Substans av Descartes: \url{https://plato.stanford.edu/entries/substance/}
    \item Encyclopaedia Britannica. (n.d.). Descartes, the pineal soul, and brain-stem death. Hentet 15/10/2025, fra \url{https://www.britannica.com/science/death/Descartes-the-pineal-soul-and-brain-stem-death}
    \item Store norske leksikon. (n.d.). Dualisme. Hentet 15/10/2025, fra \url{https://snl.no/dualisme}
    \item Wikipedia. (n.d.). \textit{Res extensa}. I \textit{Wikipedia}. Hentet 15/10/2025, fra \url{https://en.wikipedia.org/wiki/Res_extensa}
    \item Merriam-Webster. (n.d.). \textit{Res cogitans}. I \textit{Merriam-Webster}. Hentet 15/10/2025, fra \url{https://www.merriam-webster.com/dictionary/res%20cogitans}
    \item Cappelen, H., Torsen, I. \& Watzl, S. (2021). \textit{Vite, være, gjøre: Exphil}. Gyldendal
    \item Encyclopaedia Britannica. (n.d.). Vivisection. Hentet 15/10/2025, fra \url{https://www.britannica.com/science/vivisection}
\end{itemize}



\end{document}