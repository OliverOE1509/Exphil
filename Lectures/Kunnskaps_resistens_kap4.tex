\documentclass[11pt]{article}
\usepackage[a4paper,margin=1.8cm]{geometry}
\usepackage{amsmath, amssymb, amsthm}  % Essential math packages
\usepackage{graphicx}                   % For figures
\usepackage{hyperref}                   % Clickable links
\usepackage{parskip} % cleaner spacing
\usepackage{enumitem}

\newtheorem{definition}{Definisjon}

\title{Kunnskapsresistens}
\author{Oliver Ekeberg}
\date{\today}

\begin{document}
\maketitle



\tableofcontents
\vspace{1em}
\vspace{1em}
\vspace{1em}
\vspace{1em}

Wikforss bruker Kunnskapsresistens til å snakke om at mennesker ikke tilegner seg kunnskap, men som er lett tilgjengelig

For eksempel:

\begin{itemize}
    \item Det at 10 prosent av Amerikanere tror at sjokolademelk kommer fra brune kuer
\end{itemize}


Det er ikke viktig at 32 millioner amerikanere tror at sjokolademelk kommer fra brune kuer. Men det illustrerer et viktig poeng:

\vspace{1em}
Evnen til å ikke tilegne seg lett tilgjengelig kunnskap, kalles for kunnskapsresistens

Kunnskapsresistens kan være svært farlig for utviklingen i et samfunn. Det fremmer kognitiv vilje fremfor direkte fakta.

\section{Kunnskapsresistens og demokrati}

Et kunnskapsresistent demokrati vil ikke kunne leve opp til det folket faktisk trenger. Resultatet under da et valg, er at det ikke blir en rasjonell.

\vspace{1em}
Kort oppsummert, en kunnskapsrik befolkning er en viktig forutseting under et demokrati. En kunnskapsresistent befolkning vil undergrave den demokratiske prosessen.


Når det gjelder politikk, viser det seg at flere og flere velgere er kunnskapsresistente. Alle mennesker er gode på hver sine ting, og politikk handler mye om et helt samfunn som har forskjellige egenskaper og beherskningsområder. Derfor vil et parti som har sine meninger om et helt samfunn, ikke nødvendigvis ha de beste løsningene. 



\section{Kunnskapsresistens versus uvitenhet}


Som Wikforss definerer det:
\begin{itemize}
    \item Uvitenhet: Unngåelig mangel på kunnskap. Det er ufattelig mye vi aldri kan vite. For eksempel om vi er alene i universet
    \item Kunnskapsresistens: Mangel på kunnskap, som vi kan gjøre noe med
\end{itemize}


For å skjønne hva kunnskapsresistens er, må vi finne ut av hva kunnskap er


\section{Hva er kunnskap?}


Hva er det å vite noe?

Wikforss har tre betingelser for at vi skal vite noe:

\begin{enumerate}
    \item For at en person skal vite noe, må han vite at det er sant. Det er umulig å vite at Gro Harlem Brundtland var statsminister i 1982, fordi hun var ikke statsminister da. \textbf{Det som ikke er sant, kan man ikke vite}. At sannhet er en nødvendig betingelse for kunnskap, kommer klart frem hvis du leser følgende: \vspace{1em}
    A: Jeg vet at Kåre Willoch var statsminister i 1982, men det er ikke sant at Kåre Willoch var statsminister i 1982
    \item Den andre betingelsen er at du tror på kunnskapen. \textbf{Hvis man ikke tror på det, kan man ikke vite det}. 
    \vspace{1em}
    B: Jeg vet at Kåre Willoch var statsminister i 1982, men jeg tror det ikke

    \item Den tredje betingelsen til Wikforss, var at du må ha evidens. For å vite noe, må du ha en god nok grunn til å tro det. En som finner sannhet i å gjette, har ikke kunnskap. For at en person skal vite noe, må han ha evidens 
\end{enumerate}

Så kommer spørsmålet:

\vspace{1em}
Hvor mye evidens, og hvor gode grunner?


\section{Hvor gode grunner? Hvor mye evidens?}

To ting jeg vet: 

\begin{enumerate}
    \item Jeg vet at sykkelen er parkert utenfor Georg Morgenstiernes hus. Dette er sant, jeg tror det er sant, og jeg har gode grunner for å tro at det er sant: Jeg husker tydelig at jeg låste sykkelen min der
    \item Jeg vet at det er mange lemurer i Madagascar. Jeg har aldri sett en lemur i mitt liv, og jeg har aldri vært i Madagascar, men jeg har lest om lemurer i bøker og på filmer. Der står de at de bor i Madagascar 
\end{enumerate}


Det er to eksempler på kunnskap. Det første fikk jeg fra å bruke sansene og observere verden rundt meg. i det andre brukte jeg kunnskap fra andre. Hvis eksperter forteller oss noe, kan vi få kunnskap via det de forteller oss.

\vspace{1em}
I begge tilfellene, har jeg ufeilbarlig evidens. Men grunnene mine gir rom for tvil.

\vspace{1em}
Med andre ord: Vi kan altså vite noe selv om evidensen vi har er feilbarlig (Kontra Descartes argument i kapittel 1).

\textbf{Fallibilisme:} Evidensen for det vi har, kan gi rom for tvil, men det betyr ikke at vi kan vite noe som er sant


Ikke alle filosofer aksepterer Fallibilisme: 

\vspace{1em}
"Jeg vet at sykkelen min er parkert utenfor GM, men det er mulig at den er stjålet og flyttet til et helt annet sted."


\vspace{1em}
Hvis fallibilismen er korrekt, er dette et helt naturlig utsagn. Men filosofer argumenterer for at evidens må være ufeilbarlig. Disse filosofene mener at vi ikke kan vite noe med mindre vi har absolutt ufeilbarlig evidens for det.

\vspace{1em}
Det er nesten alltid mulig å finne et snev av tvil i ethvert evidens (utenom matematiske og rasjonelle evidens). Hvis dette er nok til å undergrave kunnskap, vet vi ikke noe som helst.

\vspace{1em}
Filosofer som tror på dette kalles for "skeptikere". Wikforss er ikke skeptiker, hun bygger sin teori om kunnskapsresistens på en versjon av fallibilisme.


\section{Hva kunnskapsresistens er}


Hovedårsaken til kunnskapsresistens er det Wikforss kaller for "evidensresistens"

I et nøtteskall, har ikke folk god eller nok evidens for å tro. I fallibilisme, så er det bare et krav om at det evidens ikke skal være ufeilbarlig. Det sier ikke noe om hvor mye evidens som er nok for å ha kunnskap.

\vspace{1em}
Det som er klart, er at det er viktig å ha mye informasjon og god evidens dersom man skal finne sannheten.

Wikforss mener evidensresistens har tre årsaker:

\begin{enumerate}
    \item Vi tar ikke innover oss relevant og tilgjengelig informasjon: \textbf{Vi gidder ikke å lese eller spørre de som kan mer enn oss}
    \item Vi er ikke grundige nok når vi sjekker om begrunnelsene våre holder: \textbf{Vi er ikke villig til å teste antagelsene våre.}
    \item Vi holder fast ved feilaktige antagelser, selv etter at vi har gode grunner til å forkaste dem: \textbf{Vi er irrasjonelle}
\end{enumerate}



Det forskes mye på hvorfor vi er evidensresistene. Hva kjennetegner de situasjonene der vi ikke vil endre mening, selv med ny evidens? Og når er vi villig til å ta ny informasjon i betraktning?

Et demokrati der befolkningen er i stor grad evidensresistente vil i stor grad miste legitimitet og gå under på lang sikt. 

\textbf{Hvis velgere er lett manipulerte, baserer sine politiske oppfatninger på tvilsomme kilder og stort sett har liten vilje til å sette seg grundig inn i viktige politiske spørsmål, er det vansjelig å rettferdiggjøre at demokrati er en verdifull styreform}

Tvert imot blir demokrati en svært farlig styreform, der politisk aktivisme består av å manipulere velgere som ikke handler rasjonelt.


En undersøkelse utført av YouGov i 2019, viser at Norge er på tredje plass når man måler andelen av befolkningen som nekter for at verden opplever menneskeskapte klimaendringer. USA på første og Saudi Arabia på andre. Undersøkelsen viser at mange land med størst andel klimaskeptikere, i tillegg er store oljeeksportører


\vspace{1em}
Jeg vil legge til at evidensresistens ofte handler om hva som gagner sin egen situasjon. Vi er alle egosentrerte til en viss grad, uansett hvor folkelig og hederlige vi hevder å være selv. 


\end{document}
