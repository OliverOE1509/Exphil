\documentclass[11pt, a4paper]{article}

% ===== ESSENTIAL PACKAGES =====
\usepackage{geometry}
\geometry{a4paper, margin=1.2in}
\usepackage{setspace}
\onehalfspacing

\usepackage{catchfile}

% ===== WORD COUNT COMMAND =====
\newcommand{\wordcount}{%
    \ifnum\pdf@shellescape=1%
        \immediate\write18{texcount -1 -sum Oblig1.tex > wordcount.txt}%
        \CatchFileDef{\words}{wordcount.txt}{}%
        \words%
    \else%
        [Enable -shell-escape]%
    \fi%
}

% ===== CUSTOM ENVIRONMENTS FOR PHILOSOPHICAL ARGUMENTS =====
\newenvironment{argument}{\begin{quote}\itshape\textbf{Argument: }}{\end{quote}}
\newenvironment{objection}{\begin{quote}\itshape\textbf{Innvending: }}{\end{quote}}
\newenvironment{reply}{\begin{quote}\itshape\textbf{Reply: }}{\end{quote}}

% ===== DOCUMENT BEGIN =====
\title{Kap. 7 - virkeligheten: hvordan den ser ut og hvordan den er}
\author{Oliver Ekeberg}
\date{\today}



\begin{document}
\maketitle

\tableofcontents


\section{Virkelighetens natur}


\textbf{\textit{Demokrit}}: en filosof i antikken. Han mente at mente at alt som finnes, består av uendelig mengde med små udelelige atomer, og et uendelig stort tomrom rundt de.

metafysiske spørsmål: et spørsmål om virkelighetens natur.



I kap 7 skal vi ta for oss et viktig metafysisk teori. Platons teori om formene. Platon hadde et eksempel om en hule hvor det har vært fanger hele sitt liv. Det eneste disse fangene har sett er skygger, og er oppdratt til å tro at det er virkeligheten. En av fangene slipper løs av hulen og ser at det finnes en hel annen verden, og at skyggene de så bare er projeksjoner av former de ser. Fangen kommer tilbake for å slippe løs fangene, men de nekter, fordi de er for ignorante og arrogante til å tro at det er noe annet enn skygger som er virkeligheten.

Dette kan trekkes til Sokrates, som ble dømt til døden for å utfordre den regjerende autoriteten. De ble fiendtlige når de ble utfordret på spørsmål de ikke visste svaret på. \textbf{I dette kapittelet skal vi gå dypere enn hule-analogien}


\section{Hva er et godt liv?}


\textbf{\textit{Epistemologi: Teorien om kunnskap}}

Dette er et epistemoligisk spørsmål: Hva er et godt liv?

Det er tilsynelatende noen som tror de vet. Men alle har forskjellige meninger om hva som er viktig, avhengig av egne verdier og tid. \\



"Hvor er nøkkelen min?" er et ikke epistemologisk spørsmål fordi vi vet hvordan nøkkelen vår ser ut. Dermed vet vi at vi har funnet den hvis vi finner den.\\


Si du ikke vet hva som er et godt liv. Da kan du lete etter svaret, men du vet jo ikke svaret selv, så hvordan kan du vite at du har funnet svaret?\\





\textbf{\textit{Menos paradoks}}
\begin{enumerate}
    \item hvis vi leter etter et svar, betyr det at vi ikke vet hva det er.
    \item Hvordan kan vi gjenkjenne noe vi ikke vet hva er?
    \item Hvis vi visste hva vi lette etter, ville vi ikke lett ettet det til å begynne med
\end{enumerate}



Det er to temaer Platon var interessert i:

\begin{itemize}
    \item \textbf{\textit{Å lære om dyd}}
    \item \textbf{\textit{Å lære om geometri}}
\end{itemize}


Er det slik at det å tilegne seg kunnskap, forutsetter at man allerede vet svaret på det man spør etter, slik Menos paradoks prøver å vise?\\

Hvis det er riktig, kan vi ikke tilegne oss kunnskap om noe som helst



\section{Sjelen erkjenner formene}

Platon sa at:\\
\textbf{\textit{Kunnskap om hva noe er, er ifølge Platon kunnskap om tingens form}}

Hva er så disse formene, og hvordan kan vi vite noe om dem? Dette diskuteres i den andre og tredje teksten, begge utdrag fra "Faidon"\\

Hovedtemaet i Faidon er: \textbf{\textit{Når vi dør, dør sjelen vår også?}}\\

Det er i denne sammenheng at formene og sinnets tilknytning diskuteres\\

Med platons begrep om form, går det ut på at to blyanter kan ikke ha lik lengde (Hvis vi vet at det skal mye til for at to blyanter har akkurat lik lengde med atomer). Men vi erkjenner at to blyanter har likhet om de har lik lengde. Med andre ord vet vi hva likhet betyr. Dermed vil Platons begrep om former gå ut på at vi erkjenner likheten mellom blyantene.\\

Platon mente at sjelen vår er i stand til å kommunisere direkte med formene. Vi har fått kunnskapen om at blyantene har lik lengde fra selve likhetens form.\\


\textbf{\textit{Hvordan er kunnskap om former knyttet til sjelens udødelighet?}}
Platon argumenterer for at vi ikke lærer former, men gjenkjenner dem. Det må derfor ha blitt født med oss, og når vi dør, vil da sjelen bare frigjøres fra kroppen. Dermed blir kunnskap om former et bevis på at sjelen er udødelig.

Oppsummert. 

\begin{itemize}
    \item Former = uendelige og uforanderlige
    \item Kunnskap = tilgang på formene
    \item Sjelen = må være udødelig for å gripe formene og det som er evig.
\end{itemize}



Argument som brukes i Faidon:
Vi begynner med hva vi vet. Vi vet hva likhet er. Det er derfor umulig for oss å ha kunnskap hvis all kunnskap kommer fra erfaringer. Ut fra dette, konkluderes det at det må være en anne type kunnskap. Nemlig kunnskap om former. Disse formene er ikke noe vi kan ta på, bare noe abstrakt. Det må da finnes en immateriell og evig verden med evige objekter. Sjelen vår har kunnskap om disse abstrakte objektene våre, dermed må sjelen vår være udødelig

\textbf{\textit{Sjelen må være udødelig fordi den kan gjenkjenne former som er evige og udødelige, derfor må sjelen også være udødelig}}

Dette kalles et \textit{Transcendalt argument}


\section{Formene forklarer}


\textbf{\textit{Hva er forholdet mellom formene og tingene i den virkelige verden?}}

I det siste utdraget argumenterer Platon for at formene på en eller annen måte forklarer de alminnelige formene\\

alminnelige verden := virkelige verden



\subsection{Formene og sinnet}

Noen ganger kalles platons teori for "platons idelære". På gresk, betyr form "eidos" eller idea på engelsk. Dette får det til å høres ut som at former eksisterer i sinnet vårt. På hvilken måte er former forbundet med sinnet? eksisterer former i sinnet eller i den alminnelige verden?


\section{studiespørsmål}

\begin{enumerate}
    \item Hvordan går samtalen i den første teksten fra å handle om dyd til å handle om spørsmål om matematikk? Hvordan er disse emnene knyttet sammen?
    \item Hva prøver diskusjonen med den unge gutten, som du leser i det første utdraget, å illustrere?
    \item Hva er begrunnelsen til Sokrates i den andre teksten for å tro at sjelen kjenner formene, men at den ikke kan kjenne formene fra sanselige erfaringer eller ved å ha lært dem av andre?
    \item Hvorfor syne Sokrates at "naturvitenskapelige" forklaringer på hvorfor ting er like eller små, ikke er tilfredsstillende? På hvilken måte er forklaringen ut ifra formene bedre? (Hint: se siste utdrag fra Faidon)
\end{enumerate}

\subsection{Dette bør du kunne}

\begin{itemize}
    \item Forstå og være i stand til å forklare "Menons paradoks"
    \item Kjenne den grunneleggende strukturen til Platons viktige teori om formene.
    \item Kjenne grunntrekkene i Platons beskrivelse av hvordan vi blirkjent med formene
\end{itemize}









\end{document}