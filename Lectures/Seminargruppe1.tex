\documentclass[11pt]{article}
\usepackage[a4paper,margin=1.8cm]{geometry} % change margins
\usepackage{amsmath, amssymb, amsthm}
\usepackage{graphicx}
\usepackage{hyperref}

\newtheorem{definition}{Definisjon}

\title{Exphil seminargruppe 1/6}
\author{Oliver Ekeberg}
\date{\today}

\begin{document}
\maketitle

\section{Argumenter og deres analyse}

\begin{flushleft}
    \begin{enumerate}
        \item Vitenskap er et systematisk søk etter viten eller kunnskap
        \begin{itemize}
            \item Kunnskap krever evidens eller begrunnelse
        \end{itemize}
        \item Viten er en sannhet med en velbegrunnet oppfatning
    \end{enumerate}
\end{flushleft}

\subsection{Hva er et argument?}

\begin{definition}
    \textbf{Et forsøk på å begrunne synspunkt eller oppfatning. Tenk på et argument som en invitasjon til å tenke med forfatteren}
\end{definition}

Et argument består av:
\begin{enumerate}
    \item Konklusjon - det det begrunnes for  
    \item Premiss - En påstand som skal støtte for konklusjonen
\end{enumerate}

Premissene kommer som støtte for konklusjonen.
For eksempel:
\begin{itemize}
    \item Sokrates er et menneske
    \item Alle mennesker er dødelige
    \item Sokrates er dødelig
\end{itemize} 

I dette eksemplet er premisset at alle mennesker er dødelige.
Dette er implisitt, fordi det ikke trengs en forklaring

\begin{definition}
    \textbf{Implisitt premiss: Et premiss som vi tar som allmen kunnskap. Det at alle mennesker er dødelige tar vi som implisitt}
\end{definition}

\begin{definition}
    \textbf{Eksplisitt premiss: Et premiss som er klart og tydelig formulert i en tekst.}
\end{definition}

Men hvis vi får spørsmålet, \textbf{Hvorfor det?}, må vi begrunne

\begin{itemize}
    \item Da spør vedkommende om begrunnelse for at jeg hevder min konklusjon. Da kan jeg si
    \begin{itemize}
        \item Fordi det og det
        \item Trenger ikke å angi min initielle konklusjon
    \end{itemize}
    \item Men en begrunnelse er en begrunnelse av noe spesifikt, så vi må angi konklusjonen og premissene eksplisitt    
\end{itemize}


\subsection{Hva avgjør om en tekst er et premiss, konklusjon eller ingen av delene?}

\textbf{Om en tekst er et premsiss eller konklusjon avhenger av rollen den teksen spiller i det avsnittet - og det vil til syvende og sist komme ned til forfatterens intensjon}

\textbf{Eksempler}
\begin{enumerate}
    \item "Denne banan er gul. Altså er den moden". Her er "denne banan er gul er premisset
    \item "Bananer er enten gule, grønne eller røde. Bananen i posen er verken rød eller grønn, altså er bananen gul". Her er setningen en konklusjon i et argument
    \item "Hvilken banan vil du ha? Denne banan er gul. Den vil jeg ha". Her argumenteres det ikke for noe, og setningen er verken premiss eller konklusjon
\end{enumerate}

Det er mulig å finne tekst som kunne vært et arguemnt

For eksmpel:

Å finne logiske relasjoner mellom visse setninger.

Jeg kan foreksempel si at jeg er på jernbanetorget. Om dette er sant, er det også sant at jeg er i Oslo. Men det er ikke hensikten å kommunisere dette fra forfatteren

\vspace{1em}

\textbf{Andre faktorer som gjør det vanskelig å finne arguemnt}
\begin{enumerate}
    \item Det er flere premiss i et arguemnt
    \item Et arguent kan være en hel setning
\end{enumerate}
\vspace{1em}

Ta for eksempel det første eksemplet vårt:
\vspace{1em}
\begin{enumerate}
    \item Alle mennesker er dødelige. Sokrates er et menneske. Sokrates er dødelig.
    \item Alle mennesker er døddelige og Sokrateser et menneske. Sokrates er  dødelig.
    \item Alle mennesker er dødelige og Sokrates er et menneske, altså Sokrates er dødelig
\end{enumerate}

\vspace{1em}

Her kan du ikke finne premisset ved å observere setningsstrukturen. Derfor må  du lese teksten NØYE!!!

\vspace{1em}

\textbf{Andre grunner til at det er vanskelig å finne arguemnt i en tekst:}
\begin{itemize}
    \item I en abstrakt tekst, er rekkefølgen på konklusjon og premiss forskjellige.
    \item En tekst kan inneholde flere premiss, eller andre ting enn premiss, feks et "statement".
\end{itemize}


\subsection{Signposting}

\begin{definition}
    Signposting betyr at forfatteren prøver å hjelpe leseren med å henvise han til konklusjonen
\end{definition}
\vspace{1em}

\subsubsection{Signposting 1}


\vspace{1em}

For eksempel er det ord som skal indikere at en setning skal oppfattes som en konklusjon. Dette er slike ord
\begin{enumerate}
    \item altså
    \item Derfor
    \item må
    \item kan ikke
    \item derav følger det
    \item derav kan vi slutte at...
\end{enumerate}

Han \textbf{må} være morderen. Ingen andre hadde anledning til å begå\\ mordet, og han er den eneste med fingeravtrykk på mordvåpenet

\subsubsection{Signposting 2}

Du har også ord som skal hjelpe med å begrunne en konklusjon

\begin{enumerate}
    \item av den grunn
    \item Fordi
    \item siden
    \item hvis vi antar at
    \item av den grunn
\end{enumerate}


\subsubsection{Signposting 3}

Vi har også uttrykk som peker på premisser og konklusjoner

\begin{enumerate}
    \item ... beviser at...
    \item ...viser rimeligvis at...
    \item ...kan vises ved å anta...
    \item ...gir oss grunn til å tro at...
\end{enumerate}





\end{document}
