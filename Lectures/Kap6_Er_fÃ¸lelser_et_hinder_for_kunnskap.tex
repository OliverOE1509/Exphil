\documentclass[11pt, a4paper]{article}

% ===== ESSENTIAL PACKAGES FOR PHILOSOPHY =====
\usepackage[english]{babel} % For hyphenation and linguistic rules
\usepackage{geometry} % For comfortable margins for reading
\geometry{a4paper, margin=1.2in} % Slightly larger margins for notes
\usepackage{setspace} % For line spacing
\onehalfspacing % 1.5 line spacing for readability

\usepackage[style=apa, backend=biber]{biblatex} % For APA-style citations (great for philosophy)
\addbibresource{references.bib} % Your BibTeX file

% ===== PACKAGES FOR LOGIC & ARGUMENT STRUCTURE =====
\usepackage{logicproof} % For formal logic proofs (optional)
\usepackage{enumitem} % For customizing lists

% ===== CUSTOM ENVIRONMENTS FOR PHILOSOPHICAL ARGUMENTS =====
\newenvironment{argument}{\begin{quote}\itshape}{\end{quote}} % For quoting arguments
\newenvironment{objection}{\begin{quote}\itshape\textbf{Objection: }}{\end{quote}} % For objections
\newenvironment{reply}{\begin{quote}\itshape\textbf{Reply: }}{\end{quote}} % For replies

% ===== DOCUMENT BEGIN =====
\title{Kap 6 Er følelser et hinder for kunnskap?}
\author{Oliver Ekeberg}
\date{}

\begin{document}
\maketitle


\tableofcontents

% No abstract is typical for shorter philosophy essays

\section{Oppsummering}


Jagger ville tilbakevise den vestlige myten om hva rasjonell er. 

At forskning skal være objektiv og ikke påvirket av emosjoner, vil føre til diskriminering og undergraving av kvinner innen vitenskapen.

Jagger sin definisjon på emosjon mener at man må legge mindre vekt på følelser. Definisjonen på emosjon er FVM. Altså
\begin{enumerate}
    \item Det er en følelse du får av en situasjonen
    \item Det er en vurdering av situasjonen
    \item Motivasjon til å handle for å navigere denne situasjonen
\end{enumerate}


Jagger sa at: \textbf{\textit{Hvis emosjoner motiverer handling, har innhold, innebærer vurderinger og er under vår kontroll, er det lettere å akseptere at de kan spille en rolle i vitenskapen.}}


Videre så er vitenskap allerede emosjonelt styrte. Vi forsker på det som interesserer oss. En detektiv er styrt av emosjoner i å finne den skyldige på bakgrunn av egen mistanke.

\section{Introduction}
\label{sec:introduction}

I den første delen av boken snakker vi om vitenskap og kunnskap. Men ikke om vitenskap og følelser. 

\section{Idealet om den rasjonelle og derfor ikke-rasjonelle forsker}
\label{sec:argument}

Et viktig mål for Jaggar er å tilbakevise den vestlige myten om hva det er å være rasjonell. EN viktig del av forskning forutsetter at vi setter alt av emosjoner til side.
Med andre ord; hvis forskning er påvirket av emosjoner, kan vi ikke stole på resultatet.

\vspace{1em}


La oss si at Jensen er blitt drept. Detektiven for saken konkluderer med at Berg skyldig i drapet. 
Når detektiven skal legge frem bevis, kan hen på ingen måte henvise til emosjoner. 
Det er ikke gyldig bevis for detektiven å si at "Du ser skyldig ut". Beviset må være objektivt, og det innebærer at emosjoner og følelser er lagt til side.


Ta for eksempel Einsteins relativitetsteori. Det at fysikere kan akseptere denne teorien i dag, har ingenting med emosjoner å gjøre.




\vspace{1em}

At vitenskap bær være objektiv er ubestridelig. Men Jaggar argumenterer for at det er en forenkling.
Ifølge Jaggar spiller emosjoner en sentral rolle i vår søken om kunnskap. Det er tre sentrale konklusjoner i Jaggars artikkel
\begin{enumerate}
    \item Vestens filosofi og vitenskap har betraktet emosjoner som et hinder for å finne kunnskap. Ifølge Jaggar er dette feil. Emosjoner spiller en vitkig og poitiv rolle i vitenskapelig og intellektuelt arbeid
    \item Jagger fremlegger en teori om hva emosjoner er. Den har 4 viktige deler
    \begin{itemize}
        \item Emosjoner har innhold
        \item De er aktive, ikke passive
        \item Vi kan kontrollere og styre (til en viss grad)
        \item emosjoner er sosialt konstruerte
    \end{itemize}
    \item Resultatet av den ikke-emosjonelle forsker, er "Epistemisk utestengning" av kvinner og andre grupper. Når man blir sett på som emosjonell, impliserer det, ifølge det rådende forskeridealet at man ikke er rasjonell. Dette har plasser kvinner i en epistemisk mindreverdig eller underlegen posisjon 

\end{enumerate}



\begin{objection}
    Jeg skjønner ikke Jagger sin nye teori om hva emosjoner er, og hvorfor det skal motbevise at kunnskap skal være objektiv, og at emosjoner er et hinder for dette. Dette virker som ytterlig bullshit
\end{objection}



\section{Feministik kunnskapsteori og feministisk vitenskapsteori}


Feministisk- kunnskapsteori og vitenskapsteori har en deskriptiv del og en normativ del.

\begin{itemize}
    \item Deskriptiv: hvordan kjønn påvirker hvordan vi forstår kunnskap og vår egen vitenskapelige praksis
    \item Normativ: Utvikle teorier om hvilke roller kjønn \textit{bør} spille for disse områdene.
\end{itemize}


\textbf{\textit{Elizabeth Anderson: Fem. kunnskaps og vitenskapsteori kartlegger hvordan de dominerende oppfatningene og praksisene knyttet til kunnskapsskrivelse, kunnskapsoppnåelse og -begrunnelse systematisk diskriminerer kvinner og andre undertrykte grupper, og forsøker så å endre disse oppfatningene og praksisene slik de fremmer gruppenes interesser}}

Ifl Anderson, seks måter kunnskapspraksis diskriminerer kvinner på

\begin{enumerate}
    \item Kvinner blir ekskludert fra forskning
    \item Kvinners epistemiske autoritet undergraves
    \item Kvinnelige måter å forske på blir nedvurdert
    \item Man fremmer teorier om kvinner som fremstiller dem som underlegne
    \item Fremmer teorier som gjør kvinners rolle i kunnskapstilegning usynlig
    \item Fremmer teorier som forsterker kjønnsdiskriminering
\end{enumerate}


Sentrale spørsmål for Jaggar: \textbf{\textit{Hva er forholdet mellom rasjonalitet, emosjoner og følelser? Hva er forholdet mellom kjønnsroller, emosjoner og følelser}}


\section{Hva er Emosjoner?}


Eksempler på emosjoner

\begin{enumerate}
    \item Redsel
    \item Kjærlighet
    \item Avsky
    \item Stolthet
\end{enumerate}


Ta en svømmer som møter på en hai. Svømmerens reaksjon har \textbf{\textit{tre}} elementer som er viktig for definisjonen av emosjoner

\begin{enumerate}
    \item \textbf{\textit{F}}: Hun har en \textit{følelse}, i stor grad fysiologisk
    \item \textbf{\textit{V}} Hun \textit{vurderer} sin situasjon på en ny måte. Hun opplever redsel for haien. Vi kan si at emosjonen har et innhold og er rettet mot noe
    \item \textbf{\textit{M}} Svømmeren blir \textit{motivert til å handle}. Hun har en sterk motivasjon for å komme seg vekk fra haien, eller unngå å bli spist
\end{enumerate}

Alle emosjoner kan generaliseres ned på det samme nivå (FVM). Dette er ingen fasit. Folk har forskjellige meninger.
\vspace{1em}
Jeg spør: \textbf{\textit{Hva er forskjellen på emosjon og følelse}}:

En emosjon er en fysiologisk eller psykisk reaksjon av en helhetlig opplevelse

En følelse er en indre subjektiv opplevelse av en emosjon.


\section{Jaggar mot følelsesbaserte syn på emosjoner}

Jaggar påpeker at teorier som ser på emosjoner som følelser, må akseptere at emosjoner i stor grad er utenfor vår kontroll

Hvis emosjoner skal spille en større rolle i hvordan vi får kunnskap, må vi forkaste følelses delen. Jagger har flere grunner til dette

\begin{enumerate}
    \item Vi kan ha emosjoner uten at det er ledsaget av en følelse. Vi kan være stolte, uten at det komme fra en følelse. Hvis dette er riktig, kan ikke følelser og emosjoner være det samme
    \item Emosjoner er sosialt konstruerte - de er ikke primitive og ikke kulturelt universelle. Det finnes kulturer hvor emosjoner ikke forekommer. For eksempel at et barn er stolt av å rydde rommet
    \item Jaggar vil vektlegge motivasjonskomponenten. De er aktive måter å forholde seg til virkeligheten på. Ikke passive reaksjoner
\end{enumerate}

\vspace{1em}

Jaggars teori om emosjoner fokuserer på V og M og betrakter F som mindre viktig. 
\vspace{1em}

\textbf{\textit{Hvis emosjoner motiverer handlig, har innhold, innebærer vurderinger og er under vår kontroll, er det lettere å akseptere at de kan spille en rolle i forskning.}}



Et argument for denne definisjoen er at forskning på et tema foregår, fordi det er \textbf{\textit{interesse}} for det temaet.

\vspace{1em}
Jaggar sier også at våre teorier og observasjoner er direkte påvirket av emosjoner.

For eksempel: 1800-tallets antropologer var ifølge Jaggar i stor grad styrt av forakt av ikke-hvite mennesker. Det påvirket teorier og observasjoner antropologer gjorde. 


Obervasjoner som blir brukt som evidens for teorier er i tillegg formet av emosjoner. Når vi ser eller hører noe, må det tolkes. Slike tolkninger, er delvis betinget av våre emosjonelle reaksjoner. Hva vi ser og hører er også betinget av hva vi fokuserer på, og dette er delvis påvirket av emosjonelle reaksjoner.


\textbf{\textit{Fokus er sterkt påvirket av emosjoner, og fokus påvirker hva vi observerer, og hvordan vi tolker det vi observerer}}


\section{Tilbake til detektiven og mordet på Jensen}

Dersom detektiven er sterkt mistenkelig av Berg og upålitelig, vil detkt. fokusere mer etterforskning på Berg. Vi ser at motivasjon og forskningsfokus kan påvirkes av emosjoner.


Vi kan finne to personer. En som beundrer og en som hater Berg. Vi vil da observere at begge personer får forskjellige konklusjoner, på bakgrunn av deres emosjoner.

Om Jagger har rett, får vi flere vanskelige spørsmål:

\begin{enumerate}
    \item Hvordan kan vi ha rasjonelle diskusjoner, og hvordan kan vi oppå rasjonell enighet?
    \item Hvordan kan vi komme frem til hva som vil være "korrekte" emosjoner i en rasjonell diskusjon?
\end{enumerate}



\end{document}