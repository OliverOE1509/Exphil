\documentclass[11pt]{article}
\usepackage[a4paper,margin=1.8cm]{geometry}
\usepackage{amsmath, amssymb, amsthm}  % Essential math packages
\usepackage{graphicx}                   % For figures
\usepackage{hyperref}                   % Clickable links
\usepackage{parskip} % cleaner spacing
\usepackage{enumitem}

\newtheorem{definition}{Definisjon}

\title{Kap2 Vitenskapelig Metode, Empirisme og Solflekker}
\author{Oliver Ekeberg}
\date{\today}

\begin{document}
\maketitle


\section{Empirisme}

Descartes: "Cogito ergo sum" -> "I think, and therefore I am"


I kapittel 1 lærte vi om Descartes hvor han hevdet at vi ikke kan stole på våre sanser, og er dermed utsatt for å være styrt av en ånd med onde hensikter. 
En krittikk for dette er at vi har brukt og stolt på våre sanser fra dagens morgen. Hvis vi blir jaget av en tiger, forteller 
intuisjonen vår at vi skal gjemme oss for den for å beholde livet.

Alle fremskritt i naturvitenskapen er basert på observasjoner rundt oss. For å se disse observasjonene er vi avhengig av sansene våre.


\begin{definition}
    \textbf{Empririsme:} Emprisisme er en teori som støtter opp forholdsvis åpenbare observasjoner av hvordan vi lever.
\end{definition}


Ifølge empirisme, må all kunnskap forankres i erfaring. Denne versjonen av empirisme har å gjøre med hva som gir språk mening.


\section{Empirisme versus rasjonalisme}

Empirisme er ofte kontrastert med rasjonalismen.

\begin{definition}
    \textbf{Rasjonalisme:} Fundamentet for all kunnskap er basert på vår fornuft.
\end{definition}

Du kan finne gode argumenter på begge sider. Et argument for empirisme er at vi kan tenke selv og obervere rundt oss kunnskap. 
Mens rasjonalismen har en viktig betraktning. Nemlig at naturvitenskapelig kunnskap bygger ikke på observasjon alene. 

Galileo sa at matematikken også spiller en rolle i all vitenskap. Det er vanskelig å argumenter for at matematiksk vitenskap er bygd opp av observasjoner

Hvis vi vil vite om $2+2=4$, kan vi ta 2 steiner og 2 steiner og legge de sammen. Om summen ikke blir lik 4, vet vi bare at vi har enten telt feil eller mistet en stein. 
Det sier oss ikke at $2+2 \neq 4$. Dette er et argument som støtter opp rasjonalisme

Selv i paradigmer av empirisk naturvitenskap, står matematikken sentralt. Det er vanskelig å se for seg fysikken uten matematikken. Selvom fysikken er et fag basert på observasjoner.

Et matematisk teorem er en regel som har et fornuftig og forholdsmessig intuitivt bevis. En naturlov er et teorem som er umulig å bevise.
Men vi har funnet naturlovene ut fra observasjoner. Som for eksempel Newtons andre lov: $\vec{F} = m \vec{a}$.


\section{Et paradigme på tidlig vitenskapelig arbeid: Galileos studier av solflekker}

På Galileos tid var det et geosentrisk syn som sier at alle planeter, stjerner og hele universet går rundt jorden. Det var Ptolemais som hevdet dette. Et alternativt
syn ble fremsatt av Kopernikus på 1500-tallet. Ifølge Kopernikus var solen i sentrum og jorden gikk rundt solen. Historien om hvordan vi gikk til det vi mener idag er allmenn kunnskap (heliosentrisk syn), er en historie om kombinasjonen av empirisme og rasjonalisme. Altså kombinasjonen av naturvitenskapelige observasjoner og 
rasjonaliteten bak matematikken. Summen av dette gir oss vitenskapelig innsikt


På denne tiden var det gode argument fra begge sider. Kopernikus argumenterte for at
\begin{itemize}
    \item Jorden spinner rundt sin egen akse og går rundt jorden.
\end{itemize}

Tilhengerne av det geosentriske synet påpekte at:
\begin{itemize}
    \item Hvis jorden roterer om sin egen akse, så vil ethvert punkt bevege seg i 1600km/t og hele jorden ville bevege seg over 1,6 millioner km per dag.
    \item Geosentrikerne spurte så: "Hvorfor merker vi ikke disse bevegelsene?"
\end{itemize}


\section{Teleksoper og solflekker}

Grunnen til at vi vet så mye om universet i dag er på grunnlag av observasjoner. Vi har oppfunnet teleskop og mikroskop for å se
ting som det nakne øyet ikke kan. Galileo oppfant teleskopet, som ga han evnen til å observere de andre planetene. 
Her så han at Jupiter hadde flere måner den og, i likhet med jorden. Dette var et argument han brukte for at jorden ikke var unik, og kunne derfor ikke være sentrum av universet.
Galileo oppdaget også solflekker, noe som ikke var kjent før.


\section{Galileo versus Schneier}

Scheiner hadde også oppdaget solflekker, men han mente at det var andre planeter mellom jorda og sola. Hans premiss var at flekkene bevegde på seg.

Galileo argumenterte mot Scheiner ved å si at flekkene bevegde på seg med ujevn fart. Galileo brukte geometri og tegninger for å bevise at dette stemte

Her er fem aspekter ved debatten som hjelper oss å forstå empirisk vitenskap.

\begin{enumerate}
    \item Den sentrale type observasjin er assistert av teknologisk
    \item Observasjiner må nesten alltid tolkes.
    \item Det er også mulig for Scheiner å forklare fartsvariasjoner.
    \item Som tidligere nevnt spiller matematikk en viktig rolle i Galileos argument. Ikke en ren empirisk teori
    \item Legg også merke til at verken Galileo eller Scheiner bruker abstrakte filosofiske refleksjoner om empirisme, rasjonalisme og kunnskap for å forklare sitt argument.
\end{enumerate}


\section{Vitenskap og makt}

Historien om Galileo illustrerer også hvordan gode argument kan bli ignorert fordi det ikke er forenlige med politiske makters syn. I 1616 bestemte den katolske kirken at det heliosentriske synet var absurd. I 26 år ble hans arbeid overvåket av inkvisisjonen.
Både Galileo og Kopernikus sine skrifter ble sensurert. Galileo ble i 1633 dømt til livsvarig fengsel.

\end{document}
